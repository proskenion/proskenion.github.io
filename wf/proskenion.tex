\hypertarget{abstract}{%
\subsection{Abstract}\label{abstract}}

It is a BlockChain platform that bases between the content provider and
the spectator. So, ``Proskenion'' named for the etymology of ``Prosium
Arch'' (Greek). 1. Having a high expression power by a combination of
primitive instruction sets. 2. Easy to customize incentive / consensus
algorithm. 3. Can change the incentive / consensus algorithm without
hard fork. Therefore, A decentralized and creator-based management
system can be realized with Proskenion.

\hypertarget{introduction}{%
\section{Introduction}\label{introduction}}

 昨今では、BitCoin\cite{1} や Ethereum\cite{2}
を始めとするブロックチェーン技術が台頭し空前の仮想通貨ブームを引き起こした。ブロックチェーン技術はビットコイン開発の過程で生まれ、ビットコインの取引を記録する分散型台帳を実現するための技術であった。分散型台帳技術はデータベースを一つのサーバで中央集権的に管理するのではなく複数のサーバで分散管理することで単一点障害耐性を持ったデータベースである。ブロックチェーンはそれに加え「ブロック」というデータの単位を生成し、チェーンのように連結していくことによりデータを保管する。このブロックは一つ前のブロックのハッシュ値を持っているためブロックに書き込まれたデータを改ざんするにはその場所から先頭までの全てのブロックを改ざんする必要がある。このため、ブロックチェーンでは改ざんに対する強い耐性を持っているとされている。分散台帳では複数のサーバが同一のデータを保持するために様々な合意形成アルゴリズムが考案されている。しかし、それらには未だ技術的課題がいくつも残されており実用に耐えうる代物ではなかった。代表的な問題としてスケーラビリティ、情報透過性、などがある。また、仕様変更に伴って発生するハードフォークにも問題があり
\emph{Ethereum}の合意形成アルゴリズムの移行は非常に大変な作業だった。

\hypertarget{precedent}{%
\subsection{Precedent}\label{precedent}}

 ブロックチェーン技術の応用先として最も注目されているのは金融業界である。なぜなら、既存システムと対比したときにブロックチェーン技術を用いることで減少できるコストが大きく極めて有用性が分かりやすいからである。しかし、その分金融向けブロックチェーンの競合性は高く大手企業や研究開発系ベンチャーが群雄割拠している。それに対してエンターテイメントの領域におけるブロックチェーンの利用は机上の空論の粋を出ておらずWeb3.0化を行う積極的なモチベーションに不足している。このことからエンターテイメント領域におけるブロックチェーンの利用は実用段階に達していない。そこで本プロダクト''Proskenion''はエンターテイメント領域への活用をメインの用途として構成され、様々なジャンルのコンテンツ配信に対応しさらにコンテンツの配信者を主体とできるようなシステムを目指して開発を行った。つまり、ブロックチェーン上のシステムをコンテンツの供給者が主体で扱えるような世界を創ることを目的とする。Proskenion
は一般的なブロックチェーン特有の機能である非中央集権的、非改ざん性、単一障害点耐性に加え、ブロックチェーンの運営主体を自在且つ動的且つセキュアに定義できる仕組みとプリミティブなコマンドの組み合わせで高い表現力のオブジェクトを操作できる仕組みを導入することでこれを実現しようとした。

\hypertarget{purpose}{%
\subsection{Purpose}\label{purpose}}

本プロジェクトでは、コンテンツの供給者(以下クリエータとする)のために高い表現力と合意形成アルゴリズムを汎用的に設計できるブロックチェーンを開発した。ブロックチェーン技術では、高い信頼性が求められる金融取引や重要データのやりとり等での利用が期待される技術であるが、その本質は非中央集権的な仕組みにある。ブロックチェーンシステムの最大の魅力は中央集権的な支配から解放されたシステムで人々が中〜大規模な経済を回すことができる点にある。ビットコインでは''通貨''を国家による中央集権的な支配から分散管理のもとにおくことに成功したが、この度のプロジェクトでは''デジタルコンテンツの流通''を法人による中央集権的な支配から分散管理におくことを目標にする。また、ただ分散管理をするのではコンテンツの流通の元締めが法人からマイナーになったに過ぎない。一般的なブロックチェーンではマイナーにあたる存在は大量の計算資源を持っているもの、仮想通貨を大量に保有しているものなどであり、今回実現したいクリエータ主体であるようなものとは異なる。そこでクリエータ自体をマイナーの立場におけるような合意形成とインセンティブの仕組みを設計できるブロックチェーンプラットフォームを開発した。また、配信するコンテンツやエコシステムの状況に応じて合意形成とインセンティブの仕組みを適宜変更する必要がある。これを実現するために、Proskenion
ではクリエータ自信のデータやコンテンツのメタデータを扱えるようなプリミティブな命令群と合意形成とインセンティブの仕組みを動的且つ自由且つセキュア且つ簡単に変更できる仕組みを導入した。当然、ブロックチェーンシステムには、システムの一部が故障しても障害の重大性に応じて機能を低下させながらも処理を続行するフォールトトレラント性を持つこと、外部からの侵入攻撃によってデータの
改竄を行うことが困難であることが求められる。これらを解決するために既存のブロックチェーンで採用されている技術も組み込んだ全く新しいブロックチェーンを開発した。このブロックチェーンプラットフォームは「複合型のアセットを管理できる」、「取引が改竄されることがない」、「いつでも利用できる」という要件を満たす。次章以降ではクリエータ主体のブロックチェーンシステム運用基盤プラットフォームの内部メカニズムについて解説する。

\hypertarget{conventions}{%
\section{Conventions}\label{conventions}}

本章では、Proskenion 上で使われる用語の定義を行う。まず、Proskenion
を構成する上の最小単位であるオブジェクトを定義する。

\hypertarget{primitive}{%
\subsection{Primitive}\label{primitive}}

 \emph{Primitive} な型 として Protocol-Buffers\cite{3}
で定義されている標準型を使用できる。Protocol-Buffers では標準で真偽型,
32bit符号付き整数型, 64bit符号付き整数型, 32bit符号なし整数型,
64bit符号なし整数型, 文字列型, バイト列型, 配列型, 辞書型
がサポートされている。

\hypertarget{address}{%
\subsection{Address}\label{address}}

 \emph{Address} 型は Proskenion
上に存在する内部データにアクセスするための一意な
\emph{id}を表す。\emph{Address}
はどの領域(\emph{Domain})の誰(\emph{Account})が持っているどんなデータ(\emph{Storage})かを表す
\emph{AccountName}, \emph{DomainName}, \emph{StorageName}からなり、
\texttt{AccountName@DomainName/StorageName}
と表記される。例えば、\texttt{jp} にいる \texttt{bob} が持っている
\texttt{passport} というデータを示すときは \texttt{bob@jp/passport}
と記述する。\emph{Domain} は \texttt{bob@ac.jp/passport} のように
\texttt{.} で繋げることで subdomain を定義できる。また \emph{Account}
を特定するだけであれば \emph{AccountName} と \emph{DomainName}
の情報のみで \texttt{AccountName@DomainName} という形で \emph{AccountId}
として特定できる。また、便宜上 \emph{AccountName}, \emph{DomainName},
\emph{StorageName} 全てが含まれた \emph{Address} を \emph{WalletId}
と呼ぶ。(ex. \texttt{alice@ac.jp/walllet} ) \emph{AccountName},
\emph{DomainName} のみが含まれた \emph{Address} を \emph{AccountId}
と呼ぶ。(ex. \texttt{alice@ac.jp}) \emph{StorageName} のみまたは
\emph{StorageName}と \emph{DomainName}のみが含まれた \emph{Address} を
\emph{StorageId} と呼ぶ。(ex. \texttt{/wallet}
\emph{or}\texttt{ac.jp/wallet} )

\hypertarget{signature}{%
\subsection{Signature}\label{signature}}

 \emph{Signature} は署名データを表す。\emph{Signature}
型は署名(\texttt{signature})と公開鍵(\texttt{public\_key})のペアを持つ。署名には秘密鍵によってハッシュ値を暗号化したバイト列を持つ。公開鍵には暗号化に使った秘密鍵と対となる公開鍵のバイト列を持つ。

\hypertarget{account}{%
\subsection{Account}\label{account}}

 \emph{Account} は Proskenion
上の資源を保持/操作するユーザを表す。\emph{Account} 型 は
\emph{AccountId}, \emph{AccountName}, \emph{PublicKeys}, \emph{Quorum},
\emph{Balance}, \emph{DelegatePeerId} からなる。\emph{AccountId} は
\emph{Account} を一意に特定する \emph{Address} を持つ。
\emph{AccountName} は \emph{Account}
の表記名を表す文字列を持つ。\emph{Publickeys} は \emph{Account}
の権限を行使するために使用する公開鍵のリストをバイト列の配列で持つ。\emph{Quorum}
は \emph{Account}
の権限を行使するために必要最小の署名の個数を表す32bit符号付き整数値を持つ。\emph{Balance}
は \emph{Account}
が保持している残高の量を表す64bit符号付き整数値を持つ。

\hypertarget{peer}{%
\subsection{Peer}\label{peer}}

 \emph{Peer} は Proskenion
が動作しているサーバ(ノード)の情報を表す。\emph{Peer} 型は
\emph{PeerId}, \emph{Address}, \emph{PublicKey}, \emph{Active},
\emph{Ban}からなる。 \emph{PeerId}はピアを一意に特定する
\emph{AccountId} を表す \emph{Address} を持つ。\emph{Address}
はピアにアクセスするために必要なグローバルIPアドレスを表す文字列を持つ。(メンバ変数
\emph{Address} は \emph{Address} 型とは関係ない)。\emph{Active}
はピアがネットワークに所属しているブロックチェーンと同期しているかどうかを表す真偽値を持つ。\emph{Ban}
はピアがネットワークから排斥されているかどうかを表す真偽値を持つ。

\hypertarget{storage}{%
\subsection{Storage}\label{storage}}

 \emph{Storage} は Proskenion 上で定義可能な key-value
データ構造である。\emph{Storage} は \emph{Id}, \emph{Object} を持つ。
\emph{Id} は \emph{Storage} を一意に特定するような \emph{id} を示す
\emph{Address} を持つ。\emph{Object} は \emph{Storage}
で保存しているデータ構造を表す文字列型をキー値, \emph{Object}
型をバリュー値とした辞書を持つ。

\hypertarget{command}{%
\subsection{Command}\label{command}}

 \emph{Command} は Proskenion
上の資源を更新するためのプリミティブな命令を表す。\emph{Command} 型 は
\emph{AuthorizerId}, \emph{TargetId}, \emph{Onf of command}
からなる。\emph{AuthorizerId} は \emph{Command} の執行者を示す
\emph{Address} を持つ。\emph{TargetId} は \emph{Command}の操作対象を示す
\emph{Address} を持つ。 \emph{One of Command}
は操作の内容を表す。詳細は{[}4.1 API: Write{]}を参照。

\hypertarget{transaction}{%
\subsection{Transaction}\label{transaction}}

 \emph{Transaction} は \emph{Proseknion}
の状態を変更する命令群を表す。\emph{Transaction} 型は
\emph{TransactionPayload} と \emph{Signature}
を持つ。\emph{TransactionPayload} は \emph{Transaction}
を生成した時間を示す64bit整数値 \emph{CreatedTime}
と順次実行される命令群を示す \emph{Command} 型の配列 \emph{Commands}
を持つ。\emph{Signature} には \emph{TransactionPayload} の \emph{Hash}
を秘密鍵で暗号化したものを \texttt{signature},
暗号化に使用した秘密鍵と対になる公開鍵を \texttt{public\_key} とした
\emph{Signature} を持つ。詳細は{[}4.1 API: Write{]}を参照する。

\hypertarget{block}{%
\subsection{Block}\label{block}}

 \emph{Block} は Proskenion の状態遷移の単位となる。\emph{Block} 型は
\emph{BlockPayload} と \emph{Signature} を持つ。\emph{BlockPayload}
はブロックチェーンの高さを表す64bit符号付き整数 \emph{Height},
一個前のブロックのハッシュを表すバイト列 \emph{PreBlockHash},
ブロックを生成した時間を表す64bit符号付き整数 \emph{CreatedTime},
\emph{Block}が持っている \emph{Transasction}
のリストのハッシュを持つバイト列 \emph{TxListHash}, ブロックが持つ
\emph{Transaction} を全て実行した後の Proskenion
の状態のルートハッシュを表すバイト列 \emph{WsvHash}, ブロックが持つ
\emph{Transaction}
を全て実行した後の取引履歴のルートハッシュを表すバイト列
\emph{TxHistoryHash}, ブロックが何番目の候補の \emph{Peer}
によって作られたかを示す32bit符号付き整数値 \emph{Round}
を持つ。\emph{Signature} は \emph{BlockPayload} のハッシュ値を
\emph{Peer} の秘密鍵で暗号化したものを \texttt{signature},
暗号化に使用した秘密鍵と対になる公開鍵を \texttt{public\_key} とする
\emph{Signature} を持つ。

\hypertarget{object}{%
\subsection{Object}\label{object}}

 \emph{Object} は Proskenion
上で作用する全てのデータ型を内包したデータ構造である。具体的には
Protocol-Buffers が標準で用意している型に加えて上記の\emph{Address},
\emph{Account}, \emph{Peer}, \emph{Storage}, \emph{Command},
\emph{Transaction}, \emph{Block}, そして辞書型
\emph{ObjectDict},リスト型 \emph{ObjectList} を持つ。

\hypertarget{objectdict}{%
\subsection{ObjectDict}\label{objectdict}}

 \emph{ObjectDict} は文字列型のキー値,
Object型のバリュー値を持つ辞書型のオブジェクトである。

\hypertarget{objectlist}{%
\subsection{ObjectList}\label{objectlist}}

 \emph{ObjectList} はObject型の列を持つリスト型のオブジェクトである。

\hypertarget{system-datastructure-mechanism}{%
\section{System DataStructure
Mechanism}\label{system-datastructure-mechanism}}

\hypertarget{datastructure}{%
\subsection{DataStructure}\label{datastructure}}

 \emph{DataStructure} は Proskenion
上で実装されている。プリミティブなデータ構造である。

\hypertarget{cachemap}{%
\subsubsection{CacheMap}\label{cachemap}}

 \emph{CacheMap}
は排他制御付きキャッシングデータ構造である。\emph{CacheMap} は
\emph{Set}, \emph{Get} メソッドを持つ。\emph{Set}
メソッドはハッシュ関数が定義された \emph{Object}
をキャッシュに保存する。\emph{Get}
メソッドはハッシュ値を指定すると同一のハッシュ値である \emph{Object}
を取得することができる。キャッシュアルゴリズムには簡易的な LRU
を採用している。この簡易的 LRU を便宜上 N-Stage Cache Algorithm
と呼称する。N-Srtage Cache Algorithm は \emph{N}
個の辞書を持つ。\emph{i} 番目の辞書を\emph{i-Stage Cache}
と呼び、最も最近使われたものが保存される辞書を
\emph{rerent-Stage}、最も使われたのが古い辞書を \emph{oldest-Stage}
と呼ぶ。\emph{Get} または \emph{Set} メソッドでアクセスされた要素を
\emph{recent-Stage} のキャッシュに乗せる。\emph{Set}
メソッドを実行した際にキャッシュのサイズを上回ったとき、\emph{oldest-Stage}
にキャッシュされている要素を無作為に一つ選び削除する。もし
\emph{oldest-Stage} のキャッシュが空の場合は \emph{oldest-Stage},
\emph{recent-Stage} をそれぞれインクリメントした後 \emph{N} で
\emph{mod}
をとる。これにより擬似的なLRUアルゴリズムを実現した。\emph{Set},
\emph{Get}はそれぞれ \emph{O(1)} で実行できる。(v1.0a)

\hypertarget{mapqueue}{%
\subsubsection{MapQueue}\label{mapqueue}}

 \emph{MapQueue}
は排他制御付きキャッシングキューデータ構造である。\emph{MapQueue} は
\emph{Push}, \emph{Pop}, \emph{Erase} メソッドを持つ。 \emph{Push}
メソッドは ハッシュ関数が定義された \emph{Object}を
\emph{Queue}に追加する。\emph{Pop} メソッドは
\emph{Queue}の先頭Objectを取得する。\emph{Erase}
はハッシュ値を指定して同一のハッシュ値を持つObjectをQueueから削除する。\emph{Push},
\emph{Pop}, \emph{Erase}はそれぞれならし計算量O(1)で実行できる。

\hypertarget{merkletree}{%
\subsubsection{MerkleTree}\label{merkletree}}

MerkleTree
はリストオブジェクトとその部分列のハッシュ値を計算できるデータ構造である。\emph{MerkleTree}
は \emph{Push}, \emph{Hash} メソッドを持つ。 \emph{Push}
メソッドはハッシュ関数が定義された \emph{Object}
をリストの先頭に追加する。\emph{Hash}
メソッドはリスト全体のハッシュ値を取得する。
簡易的な実装として\emph{RootHash}
としてハッシュ値の総和を持つ実装をしている。累積的にハッシュ値を計算することで
\emph{Push}, \emph{Hash} メソッドをそれぞれ \emph{O(1)}
で実行できる。ここで ハッシュ値の計算を \emph{O(1)} とする。

\hypertarget{merklepatriciatree}{%
\subsubsection{MerklePatriciaTree}\label{merklepatriciatree}}

 \emph{MerklePatriciaTree}\cite{4}
は部分木のハッシュ値を計算できる基数木データ構造である。\emph{MerklePatriciaTree}
は \emph{Search}, \emph{Find}, \emph{Upsert}, \emph{Set}, \emph{Get},
\emph{Hash} メソッドを持つ。\emph{Serach} メソッドではキー値と
接頭辞が一致している最も浅い内部ノード(部分木)を取得する。\emph{Find}
メソッドはキー値で参照した先の葉ノードを取得する。\emph{Upsert}
メソッドはキー値とバリュー値が定義されたノードを
\emph{MerklePatriciaTree} に追加または更新する。\emph{Set}
メソッドはルートハッシュを指定して木のルートを変更する。\emph{Get}
メソッドはルートハッシュ指定して内部ノード(部分木)を取得する。\emph{Hash}
メソッドでは木のルートハッシュを取得する。\emph{MerklePatriciaTree}
ではキー値を基数木の内部ノードとして且つ経路圧縮を行うことで
\emph{Search}, \emph{Find}, \emph{Upsert}, をそれぞれ
\emph{O(\textbar key\textbar)}, \emph{Set}, \emph{Get}, を \emph{O(1)},
\emph{Hash} を \emph{O(\textbar set of character of key\textbar)}
で計算できる。

\hypertarget{repository}{%
\subsection{Repository}\label{repository}}

 \emph{Repository} は Proskenion
上でデータを保存する機構ある。データを保存する形式またそのデータ構造について定義する。Proskenion
には3つの \emph{MerklePatriciaTree}
が実装されている。ブロックの列の状態を保存する \emph{BlockChain},
実行された取引の履歴を保存する \emph{TxHistory}, Proskenion
上のオブジェクトを保存する \emph{WSV(World State View)}
とそれぞれ定義する。\emph{WSV} では \emph{Address} 型の \emph{id}
によって指定された場所に \emph{Account}, \emph{Peer} または
\emph{Storage}
を保存している。また一時的に保存されるデータ機構として、\emph{Block}
に内包する \emph{Transaction} のリストを保存する \emph{TxList},
クライアントまた他ピアから受け取った \emph{Transaction} を保存する
\emph{TxQueue}, 他ピアから受け取った \emph{Block} を保存する
\emph{BlockQueue}, 他ピアから受け取った \emph{TxList} を保存する
\emph{TxListCache}, \emph{Peer} の通信経路を保存しておく
\emph{ClientCache} が実装されている。

\hypertarget{txlist}{%
\subsubsection{TxList}\label{txlist}}

 \emph{TxList} は \emph{Transaction} のリストと全体のハッシュを持つ
\emph{MerkleTree} である。\emph{TxList} は \emph{Push}, \emph{List},
\emph{Size}, \emph{Hash} メソッドを持つ。\emph{Push} メソッドは
\emph{Transaction} をリストに追加する。 \emph{List} メソッドは
\emph{Transaction} のリストを取得する。\emph{Size} は \emph{TxList}
が持っている \emph{Transaction} の数を取得する。 \emph{Hash} メソッドは
\emph{Transaction} リストの累積ハッシュを取得する。

\hypertarget{blockchain}{%
\subsubsection{BlockChain}\label{blockchain}}

 \emph{BlockChain} は \emph{Block} の列を保存する
\emph{MerklePatriciaTree} である。\emph{Block} のハッシュ値をキー値,
\emph{Block} の実体をバリュー値とおく。\emph{BlockChain} では
\emph{Next}, \emph{Get}, \emph{Append}メソッドを持つ。\emph{Next}
メソッドは \emph{Block} のハッシュ値を指定してそのBlockの次の
\emph{Block} を取得する。\emph{Get} メソッドは \emph{Block}
のハッシュ値を指定して \emph{Block} を取得する。\emph{Append} メソッドは
\emph{BlockChain} に \emph{Block} を追加する。\emph{MerklePatriciaTree}
を利用することでブロックチェーンの分岐に対応することができる。

\hypertarget{txhistory}{%
\subsubsection{TxHistory}\label{txhistory}}

 \emph{TxHistory} は取引の履歴を保存する \emph{MerklePatriciaTree}
である。\emph{TxList} のハッシュ値をキー値、 \emph{TxList}
の実体をバリュー値におく。また、\emph{Transaction}
のハッシュ値をキー値、\emph{TxList}
のハッシュ値とその添え字のペアをバリュー値におく。\emph{TxHistory} では
\emph{GetTxList}, \emph{GetTx}, \emph{Append}
メソッドを持つ。\emph{GetTxList} メソッドは \emph{TxList}
のハッシュ値を指定して \emph{TxList} を取得する。 \emph{GetTx} は
\emph{Transaction} のハッシュ値を指定して \emph{Transaction}
を取得する。 \emph{Append} は \emph{TxList} とそれが包含する
\emph{Transaction} を履歴に追加する。\emph{MerklePatriciaTree}
を利用することでブロックチェーンの分岐に対応することができる。

\hypertarget{worldstateview}{%
\subsubsection{WorldStateView}\label{worldstateview}}

 \emph{WSV} は Proskenion 上で操作する状態を保存する
\emph{MerklePatriciaTree} である。\emph{Address} を \emph{StorageName},
\emph{reverse(DomainName)},\emph{AccountName}
の順で並べたものをキー値、\emph{Account}, \emph{Peer} または
\emph{Storage} の実体をバリュー値におく。\emph{Account} を置く時の
\emph{StorageName} は \texttt{account}, \emph{Peer} を置く時の
\emph{StorageName} は \texttt{peer} とデフォルトで定めている。\emph{WSV}
は \emph{Query}, \emph{QueryAll}, \emph{Append}
メソッドが実装されている。\emph{Query} メソッドは検索対象の
\emph{Address} をキー値として \emph{Account}, \emph{Peer} または
\emph{Storage} を取得する。\emph{QueryAll} メソッドは検索対象の
\emph{Address} をキー値として接頭辞が一致している部分木から
\emph{Account}, \emph{Peer} または \emph{Storage}
のリストを取得する。\emph{Append} は追加する対象の \emph{Address}
をキー値として \emph{Account}, \emph{Peer} または \emph{Storage} を
\emph{WSV} に追加する。\emph{MerklePatriciaTree}
を利用することでブロックチェーンの分岐に対応することができる。

\hypertarget{proposaltxqueue}{%
\subsubsection{ProposalTxQueue}\label{proposaltxqueue}}

 \emph{ProposalTxQueue} はクライアントまたは他ピアから提案された
\emph{Transaction} を一時的に保存する \emph{MapQueue}
である。\emph{ProposalTxQueue} は \emph{Push}, \emph{Erase}, \emph{Pop}
メソッドが実装されている。\emph{Push} は \emph{Transaction} を指定して
\emph{ProposalTxQueue} に追加する。\emph{Erase}
は指定したハッシュ値と一致する \emph{ProposalTxQueue} 内の
\emph{Transaction} を削除する。\emph{Pop} は \emph{ProposalTxQueue}
の先頭 \emph{Transaction} を取得する。

\hypertarget{proposalblockqueue}{%
\subsubsection{ProposalBlockQueue}\label{proposalblockqueue}}

 \emph{ProposalBlockQueue} はクライアントまたは他ピアから提案された
\emph{Block} を一時的に保存する
\emph{MapQueue}である。\emph{ProposalBlockQueue} は \emph{Push},
\emph{Erase}, \emph{Pop} メソッドが実装されている。\emph{Push} は
\emph{Block} を指定して \emph{ProposalBlockQueue}
に追加する。\emph{Erase} は指定したハッシュ値と一致する
\emph{ProposalBlockQueue} 内の \emph{Block} を削除する。\emph{Pop} は
\emph{ProposalBlockQueue} の先頭 \emph{Block}を取得する。

\hypertarget{txlistcache}{%
\subsubsection{TxListCache}\label{txlistcache}}

 \emph{TxListCache} は他ピアから提案された \emph{TxList}
を一時的に保存する \emph{CacheMap} である。\emph{TxListCache} は
\emph{Set}, \emph{Get} メソッドが実装されている。\emph{Set}
メソッドは指定した \emph{TxList}を \emph{TxListCache}
に追加する。\emph{Get} メソッドは指定したハッシュ値から \emph{TxList}
を取得する。

\hypertarget{clientcache}{%
\subsubsection{ClientCache}\label{clientcache}}

 \emph{ClientCache} は他ピアとの接続情報を一時的に保存する
\emph{CacheMap} である。 \emph{ClientCache} は \emph{Set}, \emph{Get}
メソッドが実装されている。\emph{Set} メソッドは \emph{Peer}
と接続情報を指定して \emph{ClientCache} に保存する。\emph{Get}
メソッドは指定した \emph{Peer} の接続情報を取得する。

\hypertarget{repository-1}{%
\subsubsection{Repository}\label{repository-1}}

 \emph{Repository} は Proskenion
で管理しているデータ全体を総括して管理するインターフェースである。\emph{Repository}
は \emph{Top}, \emph{Me}, \emph{GetDelegatedAcounts}, \emph{Commit},
\emph{GenesisCommit}, \emph{CreateBlock}
メソッドが実装されている。\emph{Top}
メソッドは最新且つ最も信頼度の高いブロックチェーンの先頭のブロックを取得する。\emph{Me}
メソッドは自身のピア情報を取得する。\emph{GetDelegateAccounts}
はブロックの提案者として適当なアカウントのリストを取得する。\emph{Commit}
は \emph{Block} と \emph{TxList} を指定して \emph{BlockChain}
に実際にコミットする。\emph{GenesisCommit} は \emph{TxList}
を指定して全ての\emph{Transaction} を強制的に実行して高さ 0 の
\emph{Block} をコミットする。\emph{CreateBlock} は指定した
\emph{ProposalTxQueue}, 現在の \emph{Round},
現在の時刻から提案されたブロックを実行して取得する。先頭の \emph{Block}
が持っているハッシュ値から \emph{WSV}, \emph{TxHistory},
\emph{BlockChain}
の状態を復元してそれぞれ操作することができる。状態の復元は
\emph{MerklePatriciaTree} のルートノードのハッシュ値を参照するだけなので
\emph{O(1)} で実行できる。

\hypertarget{proskenion-domain-specific-language}{%
\section{Proskenion Domain Specific
Language}\label{proskenion-domain-specific-language}}

Proskenion Domain Specific Language(以下、ProSL) は Proskenion
上で実行できる DSL である。ProSL
は無限ループを避けるためにチューリング完全でない命令セットを用いている。ProSL
は Protocol-Buffers で定義されており、内部での DSL
の検証、実行もシリアライズされた Protocol-Buffers のデータを用いる。ver
1.0a では YAML\cite{5} によって書かれた ProSL を Protocol-Buffers
に変換するコードが実装されている。Proskenion 上では Protocol-Buffers
を用いて制御するため、 Procol-Buffers への変換コードさえ実装すれば YAML
以外でも ProSL を記述可能になる。

\hypertarget{design}{%
\subsection{Design}\label{design}}

ProSL は Proskenion 上で保存する時の \emph{Storage}
の設計は以下のようになっている。

\begin{verbatim}
{
  "prosl_type": "consensus" *or*"incecntive" *or*"update",
  "prosl": 0x....(byte *data*that *is*marshal *of*prosl)
}
\end{verbatim}

Proskenion を動かすには Consensus algorithm, Incentive algorithm, Update
algorithm そして Genesis Block
の4種類の設計をしなければならない。これらのそれぞれ異なるタイミングで動作する。

\hypertarget{consensus-algorithm-design}{%
\subsubsection{Consensus Algorithm
Design}\label{consensus-algorithm-design}}

ブロックの生成者を選出する仕組みを設計する。これは \emph{Block}
をコミットまたは作成する直前に実行される。コミットの前に実行される際はブロックの生成者が適切かどうかの検証に用いる。作成の前に実行される際は自分がブロックの作成者として適切かを判定するために用いる。これは
\emph{Account}のリストを返す。このリストの先頭から現在の \emph{Round}
番目のアカウントが信頼している \emph{Peer}
が正しいブロックの生成者である。

\hypertarget{incentive-algorithm-design}{%
\subsubsection{Incentive Algorithm
Design}\label{incentive-algorithm-design}}

インセンティブ付与の仕組みを設計する。これは \emph{Block}
をコミットする過程で実行される。これは \emph{Transaction} を返す。この
\emph{Transaction} は ProSL が実行された直後に強制的に実行され変更が
\emph{BlockChain} に刻まれる。

\hypertarget{update-algorithm-design}{%
\subsubsection{Update Algorithm Design}\label{update-algorithm-design}}

Consensus, Incentive, Update algorithm
を変更するための条件を設計する。これは \texttt{CheckAndCommitProsl}
という特殊なコマンドが実行された時に実行する。これは真偽値を返す。\emph{True}
が返った場合は提案された新たなアルゴリズムが適用される。

\hypertarget{genesis-block-design}{%
\subsubsection{Genesis Block Design}\label{genesis-block-design}}

高さ0の \emph{Block} で実行する命令群を設計する。これは Proskenion
の初回起動時に一度だけ実行される。これは \emph{Transaction} を返す。この
\emph{Transaction} は初回のブロックにて強制的に実行され変更が
\emph{BlockChain} に刻まれる。

// TODO \#\# ProSL Operators

ProSL の BNF は以下の通りである。この BNF では
\texttt{\textless{}variableName:\ operatorTypeName\textgreater{}}
の形式で記述する。\texttt{{[}op{]}} は Operator \texttt{op}
のリストを示す。\texttt{\{STRING:op\}} は \emph{key} が \emph{STRING}
である \emph{Operator} \texttt{op}
の辞書型を示す。\texttt{\textless{}var:op\textgreater{}?} は
\emph{Operator} \texttt{op} が \emph{optional} であることを示す。

\begin{verbatim}
Prosl := <ops:[ProslOperator]>

ProslOperator := <set:SetOperator> |
                  <if:IfOperator> |
                  <elif:ElifOperator> |
                  <else:ElseOperator> |
                  <err:ErrOperator> |
                  <require:RequireOperator> |
                  <return:ReturnOperator>
SetOperator := <var:STRING> <value:ValueOperator>
IfOperator := <cond:ConditionalFormula> <do:Prosl>
ElifOperator := <cond:ConditionalFormula> <do:Prosl>
ElseOperator := <do:Prosl>
ErrCatchOperator := <code:ERRCODE> <do:prosl>
RequireOperator := <cond:ConditionalFormula>
ReturnOperator := <return:ReturnOperator>

ValueOperator := <query:QueryOperator> |
                  <tx:TxOperator> |
                  <cmd:CommandOperator> |
                  <storage:StorageOperator> |
                  <plus:PlusOperator> |
                  <minus:TxOperator> |
                  <mult:MultipleOperator> |
                  <div:DivisionOperator> |
                  <mod:ModOperator> |
                  <and:AndOperator> |
                  <or:OrOperator> |
                  <xor:XorOperator> |
                  <concat:ConcatOperator> |
                  <valued:ValuedOperator> |
                  <indexed:IndexedOperator> |
                  <var:VariableOperator> |
                  <list:ListOperator> |
                  <map:MapOperator> |
                  <cast:CastOperator> |
                  <list_cmp:ListComprehensionOperator> |
                  <sort:SortOperator> |
                  <slice:SliceOperator> |
                  <is_defined:IsDefinedOperator> |
                  <verify:VerifyOperator> |
                  <pagerank:PageRankOperator> |
                  <len:LenOperator> |
                  <object:OBJECT>
<QueryOperator> ::= <authorizer:ValueOperator> <select:STRING> <type:OBJECTCODE>
                    <from:ValueOperator> <where:ValueOperator> <orderBy:ORDERBY> <limit:INT32>
<CommandOperator> ::= <create_account:MapOperator> |
                      <create_storage:MapOperator> | <add_balance:MapOperator>
                      <transfer_balance:MapOperator> | <add_publickeys:MapOperator>
                      <remove_remove:MapOperator> | <set_quorum:MapOperator>
                      <define_storage:MapOperator> | <create_storage:MapOperator>
                      <update_object:MapOperator> | <add_object:MapOperator>
                      <transfer_object:MapOperator> | <add_peer:MapOperator>
                      <activate_peer:MapOperator> | <suspend_peer:MapOperator>
                      <ban_peer:MapOperator> | <consign:MapOperator>
                      <check_and_commit_prosl:MapOperator> |
                      <update_storage:MapOperator>
<StorageOperator> ::= <object:MapOperator>
<PlusOperator> ::= <ops:ValueOperator+>
<MinusOperator> ::= <ops:ValueOperator+>
<MultipleOperator> ::= <ops:ValueOperator+>
<DivisionOperator> ::= <ops:ValueOperator+>
<ModOperator> ::= <ops:ValueOperator+>
<AndOperator> ::= <ops:ValueOperator+>
<OrOperator> ::= <ops:ValueOperator+>
<XorOperator> ::= <ops:ValueOperator+>
<ConcatOperator> ::= <ops:ValueOperator+>
<ValuedOperator> ::= <obj:ValueOperator> <type:OBJECTCODE> <key:STRING>
<IndexedOperator> ::= <obj:ValueOperator> <type:OBJECTCODE> <index:ValueOperator>
<ListOperator> ::= <objects:[ValueOperator]>
<MapOperator> ::= <objects:{STRING:ValueOperator}>
<ListComprehensionOperator> ::= <list:ValueOperator> <var:STRING> <if:ConditionalFormula>? <elem:ValueOperator>
<SortOperator> ::= <list:ValueOperator> <order:ORDERBY> <type:OBJECTCODE> <limit:ValueOperator>?
<SliceOperator> ::= <list:ValueOperator>? <left:ValueOperator>? <right:ValueOperator>?
<IsDefine::=> ::= <var:ValueOpeartor>
<VerifyOperator> ::= <sig:ValueOperator> <hasher:ValueOperator>
<PageRankOperator> ::= <Storages:ValueOperator> <toKey:ValueOperator> <outName:ValueOperator>
<LenOperator> ::= <list:ValueOperator>

<ConditionalFormula> ::=  <or:OrOperator> |
                          <and:AndOperator> |
                          <not:NotOperator> |
                          <eq:EqOperator> |
                          <ne:NeOperator> |
                          <gt:GtOperator> |
                          <ge:GeOperator> |
                          <lt:LtOperator> |
                          <le:LeOperator> |
                          <verify:VerifyOperator>
<NotOperator> ::= <op:ValueOperator>
<EqOperator> ::= <ops:[ValueOperator]>
<NeOperator> ::= <ops:[ValueOperator]>
<GtOperator> ::= <ops:[ValueOperator]>
<GeOperator> ::= <ops:[ValueOperator]>
<LtOperator> ::= <ops:[ValueOperator]>
<LeOperator> ::= <ops:[ValueOperator]>

<OrderByOperator> ::= <key:ValueOperator> <order:ORDERCODE>

<STRING> ::= String
<OBJECT> ::= Object
<INT32> ::= int32
<OBJECTCODE> ::= any | bool | int32 | int64 | uint32 | uint64 | string |
                bytes | address | signature | account | peer | list | dict |
                storage | command | transaction | block
<ORDERCODE> ::= DESC | ASC
\end{verbatim}

\hypertarget{system-core-mechanism}{%
\section{System Core Mechanism}\label{system-core-mechanism}}

Proskenion はパブリックブロックチェーンプラットフォームである。Ethereum
でも用いられている MerklePatriciaTree
上にトランザクションデータと状態を保存する分散型台帳基盤である。トランザクションは
Proskenion
の状態を変化させるコマンド列を持つ。クライアントはトランザクションに署名をつけて
Proskenion ピアへと送信する。Proskenion
は受け取ったトランザクションを他のピアへ伝搬して各自トランザクションを保存する。また、合意形成過程でブロックの生成者がトランザクションの列としてブロックを生成し他のピアへ伝搬する。ブロックを受け取ったピアはブロックをCommitして良いかの検証を行い妥当なブロックチェーンの先頭に繋げる。本章ではこれら一連の動作を行う
Proskenion
で実装されている主な機構であるゲート、合意形成、同期、について説明する。

\hypertarget{gate}{%
\subsection{Gate}\label{gate}}

Gate
はクライアント又は他のピアから送られてくるメッセージを受け取って処理する機構である。
Gate には APIGate, ConsensusGate, SyncGate がある。APIGate では
\emph{Transaction} を受け取る WriteRPC, Query を受け取る ReadRPC
が実装されている。APIGate で受け取った \emph{Transaction} は静的検証後
\emph{ProposalTxQueue} に挿入された後、他のピアに伝搬を行う。APIGate
で受け取った \emph{Query} は検証後、\emph{Query}
のルールに従ってデータを取得しピアの署名をつけて \emph{Query}
の送り主に返す。\emph{SyncGate}
では同期願いを受けたさいに自分の持っている主体の \emph{BlockChain}
の情報を全て送り主に送信する。同期処理は相互 Streaming 通信で行われる。

\hypertarget{consensus}{%
\subsection{Consensus}\label{consensus}}

Consensus は合意形成を行う機構である。\emph{Consensus}
では3つの処理が並列して無限ループしている。1つ目は \emph{Block}
を生成して他の \emph{Peer} に伝搬する処理である。このループでは
\emph{Block} がコミットされない限り一定時間おきに \emph{Round}
をインクリメントされて現在の \emph{Round} における \emph{Block}
生成者が自分であった場合に \emph{Block}
を生成、伝搬する。2つ目は伝搬された \emph{Block}
をコミットする処理である。このループでは \emph{ProposalBlockQueue} に
\emph{Block} が入ったイベントを受け取り \emph{Block} と \emph{TxList}
を復元し検証の結果妥当であればその \emph{Block}
をコミットする。3つ目は自ピアが \emph{Active}
でない場合にデータの同期を行う処理である。このループでは自ピアが全体と同期がとれているのかをチェックし取れていない場合は
\emph{Synchronizer} に同期リクエストを投げる。

\hypertarget{synchronizer}{%
\subsection{Synchronizer}\label{synchronizer}}

Synchronzier はデータ同期を行う機構である。Synchronizer では \emph{Peer}
を指定してその \emph{Peer} からメインの \emph{BlockChain}
のデータを取得して同期を行う。この同期処理は StreamingRPC
を用いて行う。同期願いを出された側は \emph{Block} と \emph{TxList}
を同期依頼主に送り続ける。終了条件は元のチェーンを持っている側が全てのブロックを送信しきるか、同期依頼を出した側が合意形成の過程で伝搬された
\emph{Block}
で先頭ブロックが更新されるか、エラーが発生するかのいずれかである。

\hypertarget{api}{%
\section{API}\label{api}}

各コマンドの作用などは API Documents
(\url{https://proskenion.github.io/docs}) を参照。

\hypertarget{write}{%
\subsection{Write}\label{write}}

\emph{Transaction} の設計は以下のようになっている。

\begin{verbatim}
message Transaction {
    message Payload {
        int64 createdTime = 1;
        repeated Command commands = 2;
    }
    Payload payload = 1;
    repeated Signature signatures = 2;
}
message Command {
    string authorizerId = 1;
    string targetId = 2;
    oneof command {
        CreateAccount createAccount = 3;
        AddBalance addBalance = 4;
        TransferBalance transferBalance = 5;
        AddPublicKeys addPublicKeys = 6;
        RemovePublicKeys removePublicKeys = 7;
        SetQuorum setQuorum = 8;
        DefineStorage defineStorage = 9;
        CreateStorage createStorage = 10;
        UpdateObject updateObject = 11;
        AddObject addObject = 12;
        TransferObject transferObject = 13;
        AddPeer addPeer = 14;
        ActivatePeer activatePeer = 15;
        SuspendPeer suspendPeer = 16;
        BanPeer banPeer = 17;
        Consign consign = 18;
        CheckAndCommitProsl checkAndCommitProsl = 19;
        ForceUpdateStorage forceUpdateStorage = 30;
   }
}
\end{verbatim}

\emph{Transaction} の静的検証では署名と \emph{Payload}
の中身が一致しているかをチェックする。\emph{Transaction}
の動的検証ではコマンド列の中にある \emph{AuthorizerId}
を操作するために必要な署名が揃っているかをチェックする。\emph{Transaction}
のコマンド列はACID性を備えている。

\hypertarget{read}{%
\subsection{Read}\label{read}}

\emph{Query} の設計は以下のようになっている。

\begin{verbatim}
message Query {
    message Payload {
        string authorizerId = 1;
        string select = 2;
        ObjectCode requstCode = 3;
        string fromId = 4;
        string where = 5;
        OrderBy orderBy = 6;
        int32 limit = 7;
        int64 createdTime = 8;
    }
    Payload payload = 1;
    Signature signature = 2;
}

message QueryResponse {
    Object object = 1;
    Signature signature = 2;
}
\end{verbatim}

\emph{AuthorizerId} は誰の権限で \emph{Query} を発行するかを
\emph{AccountId} で指定、\emph{FromId} は検索対象となる \emph{Address}
を指定、\emph{Select} は取得する \emph{Object}
の型を指定する。\emph{FromId} で指定した検索対象が範囲検索(
\emph{WalletId}
ではない)の場合に追加で検索クエリを設定できる。\emph{Where} は取得した
\emph{Object} のリストにフィルターをかける条件式を指定、\emph{OrderBy}
は取得したリストをソートするルールを指定、\emph{Limit}
は取得したリストの何番目までを返すかを指定する。\emph{Signature} には
\emph{Payload} を \emph{Authorizer} が署名したものを指定する。検索結果は
\emph{QueryResponse} として返ってくる。\emph{Object} は \emph{Query}
を実行した結果返ってきたオブジェクト、\emph{Signature} は \emph{Query}
を実行した \emph{Peer} が \emph{Object} を署名したものが入っている。

\hypertarget{disscussion}{%
\section{Disscussion}\label{disscussion}}

Proskenion の Usecase について考える。

\hypertarget{proof-of-creator}{%
\subsection{Proof of Creator}\label{proof-of-creator}}

 クリエータを主体とする新たな合意形成アルゴリズム ``Proof of Creator''
を提案する。仮にクリエータ同士の関係をフォロー/フォロワーを示す有効グラフで表現する。この有向グラフから
PageRank を計算して出した ``クリエータのrank''
をクリエータの信頼度とする。信頼度の上位4人によって指定されたシステム運営者達がブロックの生成を順繰りに行う。これにより、より良いクリエータがサーバを選出するクリエータが主体であるような運営システムを構築することができる。{[}Figure1{]}

\begin{figure}
\centering
\includegraphics{figures/figure1.png}
\caption{Figure1. Proof of Creator Mining}
\end{figure}

\hypertarget{selection-of-ms.contest}{%
\subsection{Selection of Ms.Contest}\label{selection-of-ms.contest}}

システムのテスト運用という意味でミスコンのような短期的な人気投票システムを行うのも面白い。ミスコンのエントリー者をクリエータと定義して彼らのインセンティブ設計が直接評価方法となり実証実験的なコンテストが開催可能であると思われる。{[}Figure2{]}

\begin{figure}
\centering
\includegraphics{figures/figure2.png}
\caption{Figure2. Ms.Contest Mining}
\end{figure}

\hypertarget{online-comic-market-platform}{%
\subsection{Online Comic Market
Platform}\label{online-comic-market-platform}}

 オンライン上でコミックマーケットのような仕組みを模擬する。Proskenion
を用いることで自治コミュニティが運営する同人即売サービスが実現できる。クリエータを同人作家と定義して作品の販売をマイニングと位置づける。評価は他のクリエータからの信頼度と販売利益と定義する。{[}Figure3{]}

\begin{figure}
\centering
\includegraphics{figures/figure3.png}
\caption{Figure3. Online comic market platform Mining}
\end{figure}

\hypertarget{conclusion}{%
\section{Conclusion}\label{conclusion}}

 Proskenion
はプリミティブな命令セットの組み合わせで高い表現力を持つパブリックブロックチェーンを実現した。また、合意形成とインセンティブの設計をハードフォーク無しに変更できる仕組みを導入/実現した。さらに、応用先としてあらゆるディジタルクリエータを対象に新たな収益源となりうる分散運営システムの例を提案した。

\hypertarget{feature-works}{%
\section{Feature works}\label{feature-works}}

 Feature works としては実証実験での利用として Proskenion
の特性を活かして実証実験で利用する形を模索している。具体的にはインセンティブと合意形成アルゴリズムを簡単に設計できるため、''どういった報酬設計''
が ``どのような結果をもたらすか''
の実験を中\textasciitilde 小規模で行うのに向いている。また、 Proskenion
が実運用されるための世界の基盤を創るための新たなブロックチェーン開発も進めたい。最近注目されているブロックチェーン開発ツールキットとして
``Substrate\cite{6}''
というものがある。これは複数のブロックチェーンを繋げる Project
の一環として作られたものであり、Polkadot\cite{7} の Project
の一つである。これを用いてスケーラビリティと相互交換性を担保する新たなブロックチェーンの開発を行う。

\hypertarget{acknowledgement}{%
\section{Acknowledgement}\label{acknowledgement}}


\begin{thebibliography}{9}
\bibitem{1} BitCoin {[}https://bitcoin.org/bitcoin.pdf{]}
\bibitem{2}  Ethereum {[}https://ethereum.github.io/yellowpaper/paper.pdf{]}
\bibitem{3}  Protocol-Buffers {[}https://developers.google.com/protocol-buffers/{]}
\bibitem{4} MerklePatriciaTree
{[}https://github.com/ethereum/wiki/wiki/Patricia-Tree{]}
\bibitem{5}  YAML {[}https://yaml.org/spec/history/2001-05-26.html{]}
\bibitem{6}  Substrate {[}https://www.parity.io/what-is-substrate/{]}
\bibitem{7}  Polkadot {[}https://polkadot.network/{]}
\end{thebibliography}
