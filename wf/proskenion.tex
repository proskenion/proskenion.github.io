\hypertarget{proskenion-blockchain-platform-that-is-able-to-design-incentiveconsensus-algorithm-dynamic-and-freedom.-draft-v1.0a}{%
\chapter{Proskenion: BlockChain platform that is able to design
incentive/consensus algorithm dynamic and freedom. Draft
v1.0a}\label{proskenion-blockchain-platform-that-is-able-to-design-incentiveconsensus-algorithm-dynamic-and-freedom.-draft-v1.0a}}

\hypertarget{author-takumi-yamashita}{%
\section{Author: Takumi Yamashita}\label{author-takumi-yamashita}}

\hypertarget{abstract}{%
\section{Abstract}\label{abstract}}

It is a BlockChain platform that bases between the content provider and
the spectator. So, ``Proskenion'' named for the etymology of ``Prosium
Arch'' (Greek). 1. Having a high expression power by a combination of
primitive instruction sets. 2. Easy to customize incentive / consensus
algorithm. 3. Can change the incentive / consensus algorithm without
hard fork. Therefore, A decentralized and creator-based management
system can be realized with Proskenion.

\hypertarget{introduction}{%
\chapter{1. Introduction}\label{introduction}}

昨今では、BitCoin や Ethereum
を始めとするブロックチェーン技術が台頭し空前の仮想通貨ブームを引き起こした。しかし、それらには未だ技術的課題がいくつも残されており実用に耐えうる代物ではなかった。代表的な問題としてスケーラビリティ、情報透過性、などがある。また、ハードフォークに伴う問題の例として
Ethereum の Proof of Work から Proof of Stake
へのアルゴリズムの移行は非常に大変な作業だった。ブロックチェーン技術の応用先としては金融業界が注目されているが、それは既存システムと対比したときにブロックチェーン技術を用いることで減少できるコストが大きく極めて有用性が分かりやすいからである。しかし、その分金融向けブロックチェーンの競合性は高く大手企業や研究開発ベンチャーが群雄割拠している時代である。それに対してエンターテイメントの領域におけるブロックチェーンの利用は机上の空論の粋を出ておらずWeb3.0化を行う積極的なモチベーションに不足している。このことからエンターテイメント領域におけるブロックチェーンの利用は実用段階に達していない。そこでProskenionはエンターテイメント領域への活用をメインの用途として構成され、様々なジャンルのコンテンツ配信に対応しさらにコンテンツの配信者を主体とできるようなものを目指して開発を行った。Proskenion
は一般的なブロックチェーン特有の非中央集権的、非改ざん性、単一障害点耐性に加え、ブロックチェーンの運営主体となるマイナーを自在且つ動的且つセキュアに定義できる仕組みとプリミティブなコマンドの組み合わせで高い表現力のオブジェクトを操作できる仕組みを導入することでこれを実現した。これにより、ブロックチェーン上のシステムをコンテンツの供給者が主体で扱えるような世界を創ることを目的とする。

\hypertarget{think}{%
\section{1.1 Think}\label{think}}

本プロジェクトでは、コンテンツの供給者(以下クリエータとする)のために高い表現力と合意形成アルゴリズムを汎用的に設計できるブロックチェーンを開発する。ブロックチェーン技術では、高い信頼性が求められる金融取引や重要データのやりとり等での利用が期待される技術であるが、その本質は非中央集権的な仕組みにある。ブロックチェーンシステムの最大の魅力は中央集権的な支配から解放されたシステムで人々が中〜大規模な経済を回すことができる点にある。ビットコインでは''通貨''を国家による中央集権的な支配から分散管理のもとにおくことに成功した。この度のプロジェクトでは''デジタルコンテンツの流通''を法人による中央集権的な支配から分散管理におくことを目標にする。また、ただ分散管理をするのではコンテンツの流通の元締めが法人から''マイナー''になったに過ぎない。一般的なブロックチェーンでは''マイナー''にあたる存在は大量の計算資源を持っているもの、仮想通貨を大量に保有しているものなどであり、今回実現したいクリエータ主体であるようなものとは異なる。そこでクリエータ自体を''マイナー''の立場におけるような合意形成とインセンティブの仕組みを設計できるブロックチェーンプラットフォームを開発した。また、配信するコンテンツやエコシステムの状況に応じて合意形成とインセンティブの仕組みを適宜変更する必要がある。これを実現するために、Proskenion
ではクリエータ自信のデータやコンテンツのメタデータを扱えるようなプリミティブな命令群と合意形成とインセンティブの仕組みを動的且つ自由且つセキュア且つ簡単に変更できるようにした。当然、ブロックチェーンシステムには、シ
ステムの一部が故障しても障害の重大性に応じて機能を低下させながらも処理を続行するフォールトトレラント性を持つこと、外部からの侵入攻撃によってデータの
改竄を行うことが困難であることが求められる。これらを解決するために既存のブロックチェーンで採用されている技術も組み込んだ全く新しいブロックチェーンを開発した。このブロックチェーンプラットフォームは以下の要件を満たす。1.
複合型のアセットを管理できる。 2. 取引が改竄されることがない。 3.
いつでも利用できる。これらの要件を満たした、クリエータ主体のブロックチェーンシステム運用基盤プラットフォームについて解説する。
\#\# 1.2 History.

\ldots{}

\hypertarget{conventions.}{%
\chapter{2. Conventions.}\label{conventions.}}

本章では、Proskenion 上で使われる用語の定義を行う。まず、Proskenion
を構成する上の最小単位であるオブジェクトを定義する。

\hypertarget{primitive}{%
\section{2.1 Primitive}\label{primitive}}

Primitive な型 として Protocol-buffers
で定義されている標準型を使用できる。Protocol-buffers では標準で真偽型,
32bit符号付き整数型, 64bit符号付き整数型, 32bit符号なし整数型,
64bit符号なし整数型, 文字列型, バイト列型, 配列型, 辞書型
がサポートされている。

\hypertarget{address}{%
\section{2.2 Address}\label{address}}

Address型は Proskenion 上に存在する内部データにアクセスするための unique
な id を表す。Address
はどの領域(Domain)の誰(Account)が持っているどんなデータ(Storage)かを表す
AccountName, DomainName, StorageName からなり、
\texttt{AccountName@DomainName/StorageName} と表記される。例えば、jp
にいる bob が持っている passport というデータを示すときは
\texttt{bob@jp/passport} と記述する。Domain は
\texttt{bob@ac.jp/passport} のように \texttt{.} で繋げることで subdomain
を定義できる。また Account を特定するだけであれば AccountName と
DomainName の情報のみで \texttt{AccountName@DomainName}
という形でAccountId として特定できる。また、便宜上 AccountName,
DomainName, StorageName 全てが含まれた Address を WalletIdと呼ぶ。(ex.
\texttt{alice@ac.jp/walllet} ) AccountName, DomainName のみが含まれた
Address を AccountId と呼ぶ。(ex. \texttt{alice@ac.jp}) StorageName
のみまたは StorageName と DomainName のみが含まれた Address をStorageId
と呼ぶ。(ex. \texttt{/wallet} or \texttt{ac.jp/wallet} )

\hypertarget{signature}{%
\section{2.1.2 Signature}\label{signature}}

Signature
は署名データを表す。Signature型は署名(\texttt{signature})と公開鍵(\texttt{public\_key})のペアを持つ。署名には秘密鍵によってHash値を暗号化したバイト列を持つ。公開鍵には暗号化に使った秘密鍵と対となる公開鍵のバイト列を持つ。

\hypertarget{account}{%
\section{2.1.3 Account}\label{account}}

Account は Proskenion 上の資源を保持/操作するユーザを表す。Account型 は
AccountId , AccountName, PublicKeys, Quorum, Balance, DelegatePeerId
からなる。AccountId は Account を一意に特定する Address を持つ。
AccountName は Account の表記名を表す文字列を持つ。Publickeys は Account
の権限を行使するために使用する公開鍵のリストをバイト列の配列で持つ。Quorum
は Account
の権限を行使するために必要最小の署名の個数を表す32bit符号付き整数値を持つ。Balance
は Account が保持している残高の量を表す 64bit符号付き整数値を持つ。

\hypertarget{peer}{%
\section{2.1.4 Peer}\label{peer}}

Peer は Proskenion が動作しているサーバ(ノード)の情報を表す。Peer型は
PeerId, Address, PublicKey, Active, Ban からなる。 PeerId
はピアを一意に特定する AccountId を表す Address を持つ。Address
はピアにアクセスするために必要なグローバルIPアドレス(とポート)を表す文字列を持つ。(ここでAddressはAddress型とは関係ない)。Active
はピアがネットワークに所属しているブロックチェーンと同期しているかどうかを表す真偽値を持つ。Ban
はピアがネットワークから排斥されているかどうかを表す真偽値を持つ。

\hypertarget{storage}{%
\section{2.1.5 Storage}\label{storage}}

Storage は Proskenion 上で定義可能な key-value データ構造である。Storage
は Id, Object を持つ。 Id は Storage を一意に特定するような id を示す
Address を持つ。Object は Storage
で保存しているデータ構造を表す文字列をkey,Object型をvalueとした辞書を持つ。

\hypertarget{command}{%
\section{2.1.6 Command}\label{command}}

Command は Proskenion
上の資源を更新するためのプリミティブな命令を表す。Command型 は
AuthorizerId, TargetId, onf of command からなる。AuthorizerId は Command
の執行者を示す Address を持つ。TargetId は Command の操作対象を示す
Address を持つ。 One of Command は操作の内容を表す。詳細は{[}4.1 API:
Write{]}を参照。

\hypertarget{transaction}{%
\section{2.1.7 Transaction}\label{transaction}}

Transaction は Proseknion の状態を変更する命令群を表す。Transaction型は
Payload と Signatureを持つ。Payload は Transaction
を生成した時間を示す64bit整数値 CreatedTime と順次実行される命令群を示す
Command型の配列 Commands を持つ。Signature には Transaction Payload の
Hash を秘密鍵で暗号化したものを signature,
暗号化に使用した秘密鍵と対になる公開鍵を public\_key とした Signature
を持つ。 詳細は{[}4.1 API: Write{]}を参照する。

\hypertarget{block}{%
\section{2.1.8 Block}\label{block}}

Block は Proskenion
の状態遷移の単位となる。Block型はPayloadとSignatureを持つ。Payload
はブロックチェーンの高さを表す64bit符号付き整数 Height,
一個前のブロックのハッシュを表すバイト列 PreBlockHash,
ブロックを生成した時間を表す64bit符号付き整数 CreatedTime, Block
が持っている Transasction のリストのハッシュを持つバイト列 TxListHash,
ブロックが持つTransactionを全て実行した後の Proskenion
の状態のルートハッシュを表すバイト列 WsvHash, ブロックが持つ Transaction
を全て実行した後のTransaction履歴のルートハッシュを表すバイト列
TxHistoryHash,
ブロックが何番目の優先度のPeerによって作られたかを示す32bit符号付き整数値Roundを持つ。Signature
は Block Payload の Hash を Peer の秘密鍵で暗号化したものをsignature,
暗号化に使用した秘密鍵と対になる公開鍵を public\_key 荷物 Signature
を持つ。

\hypertarget{object}{%
\section{2.1.9 Object}\label{object}}

Object は Proskenion
上で作用する全てのデータ型を内包したデータ構造である。具体的にはProtocolBuffers{[}{]}が標準で用意している型に加えて上記のAddress,
Account, Peer, Storage, Command, Transaction, Block, ObjectDict,
ObjectList を持つ。

\hypertarget{objectdict}{%
\section{2.1.10 ObjectDict}\label{objectdict}}

ObjectDict は key に文字列型, value に
Object型を指定した辞書型のオブジェクトである。

\hypertarget{objectlist}{%
\section{2.1.11 ObjectList}\label{objectlist}}

ObjectList は key に文字列型, value に
Object型を指定したリスト型のオブジェクトである。

\hypertarget{system-data-structure-mechanism}{%
\chapter{3. System Data Structure
Mechanism}\label{system-data-structure-mechanism}}

\hypertarget{datastructure}{%
\section{\texorpdfstring{3.1
\textbf{DataStructure}}{3.1 DataStructure}}\label{datastructure}}

DataStructure は
Proskenion上で実装されている。プリミティブなデータ構造である。
\textbf{3.1.1 CacheMap} CacheMap
は排他制御付きキャッシュデータ構造である。CacheMap は Set method, Get
method を持つ。Set method は Hash 関数が定義された Object
をキャッシュに保存する。Get method
はハッシュ値を指定すると同一のHash値であるObjectを取得することができる。キャッシュアルゴリズムには簡易的なLRUを採用している。この簡易的LRUを便宜上
N-Stage Cache Algorithm と呼称する。N-Srtage Cache Algorithm
はN個の辞書を持つ。i番目の辞書をi-Stage Cache
と呼び、最も最近使われたものが保存される辞書を
rerent-Stage、最も使われたのが古い辞書を oldest-Stage と呼ぶ。Get or Set
method でアクセスされた要素をrecent-Stage のキャッシュに乗せる。Set
method の際にCacheSizeを上回ったとき、oldest-Stage
にキャッシュされている要素を無作為に一つ選び削除する。もし oldest-Stage
のキャッシュが無い場合は oldest-Stage, recent-Stage
をそれぞれインクリメントした後Nで mod
をとる。これにより擬似的なLRUアルゴリズムを実現した。Set, Get
はそれぞれO(1)で実行できる。(v1.0a) \textbf{3.1.2 MapQueue} MapQueue
は排他制御付きキャッシングQueueデータ構造である。MapQueue は Push, Pop,
Erase method を持つ。 Push Method は Hash 関数が定義された Object を
Queue に追加する。Pop Method は Queue の先頭Objectを取得する。Erase
はハッシュ値を指定して同一のハッシュ値を持つObjectをQueueから削除する。Push,
Pop, Erase はそれぞれならし計算量O(1)で実行できる。 \textbf{3.1.3
MerkleTree} MerkleTree
はリストオブジェクトとその部分列のハッシュ値を計算できるデータ構造である。MerkleTree
は Push, Hash method を持つ。 Push Method は Hash 関数が定義された
Object をリストの先頭に追加する。Hash Method
はリスト全体のハッシュ値を取得する。
簡易的な実装としてRootHashとしてハッシュ値の総和を持つ実装をしている。累積的にハッシュ値を計算することで
Push, Hash をそれぞれ O(1) で実行できる。ここで
Hash値の計算をO(1)とする。 \textbf{3.1.4 MerklePatriciaTree}
MerklePatriciaTree
は部分木のハッシュ値を計算できる基数木データ構造である。MerklePatriciaTree
は Search, Find, Upsert, Set, Get, Hash Method を持つ。Serach Method
では、key と
接頭辞が一致している最も浅い内部ノード(部分木)を取得する。Find method
は key で参照した先の葉ノードを取得する。Upsert method は key と value
が定義されたノードを MerklePatriciaTree に追加または更新する。Set method
はルートハッシュを指定して木のルートを変更する。Get method
はルートハッシュ指定して内部ノード(部分木)を取得する。Hash method
では木のルートハッシュを取得する。Merkle Patricia Tree では key
を基数木の内部ノードとして且つ経路圧縮を行うことで Search, Find, Upsert,
をそれぞれ O(\textbar key\textbar), Set, Get, をO(1), Hash を
O(\textbar set of character of key\textbar) で計算できる。

\hypertarget{repository}{%
\section{3.2 Repository}\label{repository}}

Repository は Proskenion
上でデータを保存する機構ある。データを保存する形式またそのデータ構造について定義する。Proskenion
には3つの Merkle Patricia Tree{[}1{]}
が実装されている。BlockChain状態を保存する ``BlockChain'', 実行された
Transaction の状態を保存する ``TxHistory'', Proskenion
上のオブジェクトを保存する ``WSV(World State View)''
とそれぞれ呼称する。WSV では Address型のidによって指定された場所に
Account, Peer または Storage
を保存している。また一時的に保存されるデータ機構として、Blockに内包するTransactionのリストを保存するTxList,
クライアントまた他Peerから受け取った Transaction を保存する TxQueue,
他Peerから受け取ったBlockを保存するBlockQueue,
他Peerから受け取ったTxList を保存する TxListCache,
Peerの通信経路を保存しておくClientCache が実装されている。 \textbf{3.2.1
TxList} TxList は Transaction のリストを保存する MerkleTree
である。TxList は Push, List, Size, Hash Method を持つ。Push Method は
Transaction をリストに追加する。 List Method は Transaction
のリストを取得する。Size は TxList が持っている Transaction
の数を取得する。 Hash Method は Transaction
リストの累積ハッシュを取得する。 \textbf{3.2.2 BlockChain} BlockChain は
Block の列を保存する MerklePatriciaTree である。Block
のハッシュを内部ノード, Block の実態を葉ノードにおく。BlockChain では
Next, Get, Append Method を持つ。Next Method は Block の Hash
を指定してそのBlockの次のBlockを取得する。Get Method は Block の Hash
を指定して Block を取得する。Append Method は BlockChain に Block
を追加する。MerklePatriciaTree を利用することで BlockChain
の分岐に対応することができる。 \textbf{3.2.3 TxHistory} TxHistory は
Transaction の履歴を保存する MerklePatriciaTree である。TxList
のハッシュ値を内部ノード、 TxList
の実態を葉ノードにおく。また、Transaction
のハッシュ値を内部ノード、TxList のハッシュ値とその index
を葉ノードにおく。TxHistory では GetTxList, GetTx, Append Method
を持つ。GetTxList Method は TxList のハッシュを指定して TxList
を取得する。 GetTx は Transaction のハッシュを指定して Transaction
を取得する。 Append は TxList とその内部の Transaction
を履歴に追加する。MerklePatriciaTree を利用することで BlockChain
の分岐に対応することができる。 \textbf{3.2.4 WorldStateView}
WorldStateView(以下、WSV)は Proskenion 上で操作する状態を保存する
MerklePatriciaTree である。Address を
StorageName,reverse(DomainName),AccountName
の順で並べたものを内部ノード、Account, Peer または Storage
を葉ノードにおく。Account を置く時の StorageName は ``account'', Peer
を置く時の StorageName は ``peer'' とデフォルトで定めている。WSV は
Query, QueryAll, Append Method が実装されている。Query method
は検索対象の Address を指定して Account, Peer または Storage
を取得する。QueryAll method は検索対象の Address
を指定してたどった部分木の葉ノードのリストから Account , Peer または
Storage のリストを取得する。Append は追加する対象の Address を指定して
Account, Peer または Storage を WSV に追加する。 \textbf{3.2.5
ProposalTxQueue} ProposalTxQueue
はクライアントまたは他Peerから提案された Transaction を一時的に保存する
MapQueue である。ProposalTxQueue は Push, Erase Pop Method
が実装されている。Push は Transaction を指定して ProposalTxQueue
に追加する。Erase は指定したハッシュ値と一致する ProposalTxQueue 内の
Transaction を削除する。Pop は ProposalTxQueue の先頭 Transaction
を取得する。 \textbf{3.2.6 ProposalBlockQueue} ProposalBlockQueue
はクライアントまたは他Peerから提案された Block を一時的に保存する
MapQueue である。ProposalBlockQueue は Push, Erase Pop Method
が実装されている。Push は Block を指定して ProposalBlockQueue
に追加する。Erase は指定したハッシュ値と一致する ProposalBlockQueue 内の
Block を削除する。Pop は ProposalBlockQueue の先頭 Block を取得する。
\textbf{3.2.7 TxListCache} TxListCache は他Peerから提案された TxList
を一時的に保存する CacheMapである。TxListCache は Set, Get Method
が実装されている。Set method は指定した TxList を TxListCache
に追加する。Get method は指定したハッシュ値から TxList を取得する。
\textbf{3.2.8 ClientCache} ClientCache
は他Peerとの接続情報を一時的に保存する CacheMap である。 ClientCache は
Set, Get method が実装されている。Set method は Peer
と接続情報を指定して ClientCache に保存する。Get method は指定した Peer
の接続情報を取得する。 \textbf{3.2.9 Repository} Repository は
BlockChain 全体を総括して管理するインターフェースである。Repository は
Top, Me, GetDelegatedAcounts, Commit, GenesisCommit, CreateBlock method
が実装されている。Top method
は最新且つ最も信頼度の高いブロックチェーンの先頭のブロックを取得する。Me
method は自身のPeer情報を取得する。GetDelegateAccounts
はブロックの提案者として適当なAccountのリストを取得する。Commit は Block
と TxList を指定して BlockChain に実際にCommitする。GenesisCommit は
TxList
を指定して全てのTransactionを強制的に実行して高さ0のBlockをCommitする。CreateBlock
は指定した ProposalTxQueue, 現在の Round,
現在の時刻から提案されたブロックを実行して取得する。Top Block
が持っているハッシュ値から WSV, TxHistory, BlockChain
の状態を復元してそれぞれ操作することができる。状態の復元は
MerklePatriciaTree のルートノードのハッシュ値を参照するだけなので O(1)
で実行できる。

\hypertarget{proskenion-domain-specific-language}{%
\chapter{3. Proskenion Domain Specific
Language}\label{proskenion-domain-specific-language}}

Proskenion Domain Specific Language(以下、ProSL) は Proskenion
上で実行できる DSL である。ProSL
は無限ループを避けるためにチューリング完全でない命令セットを用いている。ProSL
は Protocol-Buffers
で定義されており、内部でのDSLの検証、実行もシリアライズされた
Protocol-Buffers のデータを用いる。ver1.0a では YAML によって書かれた
ProSL を Protocol-Buffers に変換する Convertor
が実装されている。Proskenion 上では Protocol-Buffers
を用いるため、Convertor さえ実装すれば YAML 以外でも ProSL
を記述可能になる。

\hypertarget{design}{%
\section{3.1 Design}\label{design}}

ProSL は Proskenion 上で保存する時の Storage
の設計は以下のようになっている。

\begin{verbatim}
{
  "prosl_type": "consensus" or "incecntive" or "update",
  "prosl": 0x....(byte data that is marshal of prosl)
}
\end{verbatim}

Proskenion を動かすには Consensus algorithm, Incentive algorithm, Update
algorithm そして Genesis Block
の4種類の設計をしなければならない。これらのそれぞれ異なるタイミングで動作する
ProSL である。 \textbf{3.1.1 Consensus Algorithm Design}
Blockの生成者を選出する仕組みを設計する。これはBlockをCommitまたはCreateする直前に実行される。これは
Account
のリストを返す。このリストの先頭から現在のRound番目のアカウントが信頼しているPeerが正しいブロックの生成者である。
\textbf{3.1.2 Incentive Algorithm Design}
インセンティブ付与の仕組みを設計する。これはBlockをCommitする過程で実行される。これは
Transaction を返す。この Transaction は ProSL
が実行された直後に強制的に実行され変更が BlockChain に刻まれる。
\textbf{3.1.3 Update Algorithm Design} Consensus, Incentive, Update
algorithm を変更するための条件を設計する。これは
\texttt{CheckAndCommitProsl}
という特殊なコマンドが実行された時に実行する。これは真偽値を返す。True
が返った場合は提案された新たなアルゴリズムが適用される。 \textbf{3.1.4
Genesis Block Design}
高さ0のBlockに刻まれるTransactionを設計する。これは Proskenion
の初回起動時に一度だけ実行される。これは Transaction を返す。この
Transaction は初回のブロックにて強制的に実行され変更が BlockChain
に刻まれる。

\hypertarget{prosl-operators}{%
\section{3.2 ProSL Operators}\label{prosl-operators}}

ProSL の BNF は以下の通りである。この BNF では
\texttt{\textless{}variableName:\ operatorTypeName\textgreater{}}
の形式で記述する。\texttt{{[}op{]}}は Operator \texttt{op}
のリストを示す。\texttt{\{STRING:op\}} は key が STRING である Operator
\texttt{op} の辞書型を示す。\texttt{\textless{}var:op\textgreater{}?} は
Operator \texttt{op} が optional であることを示す。

\begin{verbatim}
Prosl ::= <ops:[ProslOperator]>

ProslOperator ::= <set:SetOperator> |
                  <if:IfOperator> |
                  <elif:ElifOperator> |
                  <else:ElseOperator> |
                  <err:ErrOperator> |
                  <require:RequireOperator> |
                  <return:ReturnOperator>
SetOperator ::= <var:STRING> <value:ValueOperator>
IfOperator ::= <cond:ConditionalFormula> <do:Prosl>
ElifOperator ::= <cond:ConditionalFormula> <do:Prosl>
ElseOperator ::= <do:Prosl>
ErrCatchOperator ::= <code:ERRCODE> <do:prosl>
RequireOperator ::= <cond:ConditionalFormula>
ReturnOperator ::= <return:ReturnOperator>

ValueOperator ::= <query:QueryOperator> |
                  <tx:TxOperator> |
                  <cmd:CommandOperator> |
                  <storage:StorageOperator> |
                  <plus:PlusOperator> |
                  <minus:TxOperator> |
                  <mult:MultipleOperator> |
                  <div:DivisionOperator> |
                  <mod:ModOperator> |
                  <and:AndOperator> |
                  <or:OrOperator> |
                  <xor:XorOperator> |
                  <concat:ConcatOperator> |
                  <valued:ValuedOperator> |
                  <indexed:IndexedOperator> |
                  <var:VariableOperator> |
                  <list:ListOperator> |
                  <map:MapOperator> |
                  <cast:CastOperator> |
                  <list_cmp:ListComprehensionOperator> |
                  <sort:SortOperator> |
                  <slice:SliceOperator> |
                  <is_defined:IsDefinedOperator> |
                  <verify:VerifyOperator> |
                  <pagerank:PageRankOperator> |
                  <len:LenOperator> |
                  <object:OBJECT>
<QueryOperator> ::= <authorizer:ValueOperator> <select:STRING> <type:OBJECTCODE>
                    <from:ValueOperator> <where:ValueOperator> <orderBy:ORDERBY> <limit:INT32>
<CommandOperator> ::= <create_account:MapOperator> |
                      <create_storage:MapOperator> | <add_balance:MapOperator>
                      <transfer_balance:MapOperator> | <add_publickeys:MapOperator>
                      <remove_remove:MapOperator> | <set_quorum:MapOperator>
                      <define_storage:MapOperator> | <create_storage:MapOperator>
                      <update_object:MapOperator> | <add_object:MapOperator>
                      <transfer_object:MapOperator> | <add_peer:MapOperator>
                      <activate_peer:MapOperator> | <suspend_peer:MapOperator>
                      <ban_peer:MapOperator> | <consign:MapOperator>
                      <check_and_commit_prosl:MapOperator> |
                      <update_storage:MapOperator>
<StorageOperator> ::= <object:MapOperator>
<PlusOperator> ::= <ops:ValueOperator+>
<MinusOperator> ::= <ops:ValueOperator+>
<MultipleOperator> ::= <ops:ValueOperator+>
<DivisionOperator> ::= <ops:ValueOperator+>
<ModOperator> ::= <ops:ValueOperator+>
<AndOperator> ::= <ops:ValueOperator+>
<OrOperator> ::= <ops:ValueOperator+>
<XorOperator> ::= <ops:ValueOperator+>
<ConcatOperator> ::= <ops:ValueOperator+>
<ValuedOperator> ::= <obj:ValueOperator> <type:OBJECTCODE> <key:STRING>
<IndexedOperator> ::= <obj:ValueOperator> <type:OBJECTCODE> <index:ValueOperator>
<ListOperator> ::= <objects:[ValueOperator]>
<MapOperator> ::= <objects:{STRING:ValueOperator}>
<ListComprehensionOperator> ::= <list:ValueOperator> <var:STRING> <if:ConditionalFormula>? <elem:ValueOperator>
<SortOperator> ::= <list:ValueOperator> <order:ORDERBY> <type:OBJECTCODE> <limit:ValueOperator>?
<SliceOperator> ::= <list:ValueOperator>? <left:ValueOperator>? <right:ValueOperator>?
<IsDefine::=> ::= <var:ValueOpeartor>
<VerifyOperator> ::= <sig:ValueOperator> <hasher:ValueOperator>
<PageRankOperator> ::= <Storages:ValueOperator> <toKey:ValueOperator> <outName:ValueOperator>
<LenOperator> ::= <list:ValueOperator>

<ConditionalFormula> ::=  <or:OrOperator> |
                          <and:AndOperator> |
                          <not:NotOperator> |
                          <eq:EqOperator> |
                          <ne:NeOperator> |
                          <gt:GtOperator> |
                          <ge:GeOperator> |
                          <lt:LtOperator> |
                          <le:LeOperator> |
                          <verify:VerifyOperator>
<NotOperator> ::= <op:ValueOperator>
<EqOperator> ::= <ops:[ValueOperator]>
<NeOperator> ::= <ops:[ValueOperator]>
<GtOperator> ::= <ops:[ValueOperator]>
<GeOperator> ::= <ops:[ValueOperator]>
<LtOperator> ::= <ops:[ValueOperator]>
<LeOperator> ::= <ops:[ValueOperator]>

<OrderByOperator> ::= <key:ValueOperator> <order:ORDERCODE>

<STRING> ::= String
<OBJECT> ::= Object
<INT32> ::= int32
<OBJECTCODE> ::= any | bool | int32 | int64 | uint32 | uint64 | string |
                bytes | address | signature | account | peer | list | dict |
                storage | command | transaction | block
<ORDERCODE> ::= DESC | ASC
\end{verbatim}

//TODO

\hypertarget{system-core-mechanism}{%
\chapter{3. System Core Mechanism}\label{system-core-mechanism}}

Proskenion はパブリックブロックチェーンプラットフォームである。Ethereum
でも用いられている Merkle Patricia Tree
上にトランザクションデータと状態を保存する分散型台帳基盤である。トランザクションは
Proskenion
の状態を変化させるコマンド列を持つ。クライアントはトランザクションに署名をつけて
Proskenion ピアへと送信する。Proskenion
は受け取ったトランザクションを他のピアへ伝搬して各自トランザクションを保存する。また、合意形成過程でブロックの生成者がトランザクションの列としてブロックを生成し他のピアへ伝搬する。ブロックを受け取ったピアはブロックをCommitして良いかの検証を行い妥当なブロックチェーンの先頭に繋げる。本章ではこれら一連の動作を行う
Proskenion
で実装されている主な機構であるゲート、合意形成、同期、について説明する。

\hypertarget{gate}{%
\section{2.1 Gate}\label{gate}}

Gate
はクライアント又は他のピアから送られてくるメッセージを受け取って処理する機構である。Gate
には API Gate, Consensus Gate, Sync Gate がある。API Gate では
Transaction を受け取る Write RPC, Query を受け取る Read RPC
が実装されている。API Gate で受け取った Transaction は静的検証後
ProposalTxQueue に挿入された後、他のピアに伝搬を行う。API Gate
で受け取った Query は検証後、Query
のルールに従ってデータを取得しピアの署名をつけて Query
の送り主に返す。Sync Gate
では同期願いを受けたさいに自分の持っているメインの BlockChain
の情報を全て送り主に送信する。Sync は相互 Streaming 通信で行われる。

\hypertarget{consensus}{%
\section{2.1 Consensus}\label{consensus}}

Consensus は合意形成を行う機構である。Consensus
では3つの処理が並列して無限ループしている。1つ目は Block
を生成して他のPeerに伝搬する処理である。このループでは Block が Commit
されない限り一定時間おきに Round
をインクリメントされて現在のRoundにおける Block
生成者が自分であった場合に Block を生成、伝搬する。2つ目は伝搬された
Block を Commit する処理である。このループでは ProposalBlockQueue に
Block が入ったイベントを受け取り Block とTxList
を復元し検証の結果妥当であればその Block を Commit する。3つ目は自Peer
が Active でない場合にデータの同期を行う処理である。このループでは自Peer
が全体と同期がとれているのかをチェックし取れていない場合は Synchronizer
に同期リクエストを投げる。

\hypertarget{synchronizer}{%
\section{2.2 Synchronizer}\label{synchronizer}}

Synchronzier はデータ同期を行う機構である。Synchronizer では Peer
を指定してその Peer からメインの BlockChain
のデータを取得して同期を行う。この同期処理は Streaming RPC
を用いて行う。同期願いを出された側は Block と TxList
を同期依頼主に送り続ける。終了条件は元のチェーンを持っている側が全てのブロックを送信しきるか、同期依頼を出した側が合意形成の過程で伝搬されたBlockで先頭ブロックが更新されるか、エラーが発生するかのいずれかである。

\hypertarget{api}{%
\chapter{4. API}\label{api}}

各コマンドの作用などは Documents
を参照{[}https://github.com/proskenion.github.io{]}。

\hypertarget{write}{%
\section{4.1 Write}\label{write}}

Transaction in Commands の API Documents. Transaction
の設計は以下のようになっている。

\begin{verbatim}
message Transaction {
    message Payload {
        int64 createdTime = 1;
        repeated Command commands = 2;
    }
    Payload payload = 1;
    repeated Signature signatures = 2;
}
message Command {
    string  authorizerId = 1;
    string targetId = 2;
    oneof command {
        CreateAccount createAccount = 3;
        AddBalance addBalance = 4;
        TransferBalance transferBalance = 5;
        AddPublicKeys addPublicKeys = 6;
        RemovePublicKeys removePublicKeys = 7;
        SetQuorum setQuorum = 8;
        DefineStorage defineStorage = 9;
        CreateStorage createStorage = 10;
        UpdateObject updateObject = 11;
        AddObject addObject = 12;
        TransferObject transferObject = 13;
        AddPeer addPeer = 14;
        ActivatePeer activatePeer = 15;
        SuspendPeer suspendPeer = 16;
        BanPeer banPeer = 17;
        Consign consign = 18;
        CheckAndCommitProsl checkAndCommitProsl = 19;
        ForceUpdateStorage forceUpdateStorage = 30;
   }
}
\end{verbatim}

Transaction の Verify では署名とPayload
の中身が一致しているかをチェックする。Transaction の Validate
ではコマンド列の中にある authorizerId
を操作するために必要な署名が揃っているかをチェックする。Transaction
のコマンド列はACID性を備えている。

\hypertarget{read}{%
\section{4.2 Read}\label{read}}

Query の設計は以下のようになっている。

\begin{verbatim}
message Query {
    message Payload {
        string authorizerId = 1;
        string select = 2;
        ObjectCode requstCode = 3;
        string fromId = 4;
        string where = 5;
        OrderBy orderBy = 6;
        int32 limit = 7;
        int64 createdTime = 8;
    }
    Payload payload = 1;
    Signature signature = 2;
}

message QueryResponse {
    Object object = 1;
    Signature signature = 2;
}
\end{verbatim}

AuthorizerId は誰の権限で Query を発行するかを AccountId で指定、FromId
は検索対象となる Address を指定、Select は取得する Object
の型を指定する。FromId
で指定した検索対象が範囲検索(WalletIdではない)の場合に追加で検索クエリを設定できる。Where
は取得した Object のリストにフィルターをかける条件式を指定、OrderBy
は取得したリストをソートするルールを指定、Limit
は取得したリストの何番目までを返すかを指定する。Signature には Payload
を Authorizer が署名したものを指定する。検索結果は QueryResponse
として返ってくる。Object は Query
を実行した結果返ってきたオブジェクト、Signature は Query を実行した Peer
が Object を署名したものが入っている。

\hypertarget{disscussion}{%
\chapter{4. Disscussion}\label{disscussion}}

Proskenion の Usecase について考える。

\hypertarget{proof-of-creator}{%
\section{4.1 Proof of Creator}\label{proof-of-creator}}

 クリエータを主体とする新たな合意形成アルゴリズム ``Proof of Creator''
を提案する。仮にクリエータ同士の関係をフォロー/フォロワーを示す有効グラフで表現する。この有向グラフから
PageRank を計算して出した ``クリエータのrank''
をクリエータの信頼度とする。信頼度の上位4人によって指定されたシステム運営者達がブロックの生成を順繰りに行う。これにより、より良いクリエータがサーバを選出するクリエータが主体であるような運営システムを構築することができる。{[}Figure1{]}

\begin{figure}
\centering
\includegraphics{https://d2mxuefqeaa7sj.cloudfront.net/s_FA1D093D0EC7FFF24C90E55F2C15C3428DC2EF11A4A38F58350CFC712E4792F4_1551634997921_+2019-03-04+02.42.48.png}
\caption{Figure1. Proof of Creator Mining}
\end{figure}

\hypertarget{selection-of-ms.contest}{%
\section{4.2 Selection of Ms.Contest}\label{selection-of-ms.contest}}

システムのテスト運用という意味でミスコンのような短期的な人気投票システムを行うのも面白い。ミスコンのエントリー者をクリエータと定義して彼らのインセンティブ設計が直接評価方法となり実証実験的なコンテストが開催可能であると思われる。{[}Figure2{]}

\begin{figure}
\centering
\includegraphics{https://d2mxuefqeaa7sj.cloudfront.net/s_FA1D093D0EC7FFF24C90E55F2C15C3428DC2EF11A4A38F58350CFC712E4792F4_1551634998114_+2019-03-04+02.43.09.png}
\caption{Figure2. Ms.Contest Mining}
\end{figure}

\hypertarget{online-comic-market-platform}{%
\section{4.3 Online Comic Market
Platform}\label{online-comic-market-platform}}

 オンライン上でコミックマーケットのような仕組みを模擬する。Proskenion
を用いることで自治コミュニティが運営する同人即売サービスが実現できる。クリエータを同人作家と定義して作品の販売をマイニングと位置づける。評価は他のクリエータからの信頼度と販売利益と定義する。{[}Figure3{]}

\begin{figure}
\centering
\includegraphics{https://d2mxuefqeaa7sj.cloudfront.net/s_FA1D093D0EC7FFF24C90E55F2C15C3428DC2EF11A4A38F58350CFC712E4792F4_1551634998118_+2019-03-04+02.42.58.png}
\caption{Figure3. Online comic market platform Mining}
\end{figure}

\hypertarget{conclusion}{%
\chapter{5. Conclusion}\label{conclusion}}

 Proskenion
はプリミティブな命令セットの組み合わせで高い表現力を持つパブリックブロックチェーンを実現した。また、合意形成とインセンティブの設計をハードフォーク無しに変更できる仕組みを導入/実現した。さらに、応用先としてあらゆるディジタルクリエータを対象に新たな収益源となりうる分散運営システムの例を提案した。

\hypertarget{feature-works}{%
\chapter{6. Feature works}\label{feature-works}}

 Feature works としては実証実験での利用として Proskenion
の特性を活かして実証実験で利用する形を模索している。具体的にはインセンティブと合意形成アルゴリズムを簡単に設計できるため、''どういった報酬設計''
が ``どのような結果をもたらすか''
の実験を中\textasciitilde 小規模で行うのに向いている。また、 Proskenion
が実運用されるための世界の基盤を創るための新たなブロックチェーン開発も進めたい。最近注目されている
BlockChain 開発ツールキットとして ``Substrate''
というものがある。これは複数のブロックチェーンを繋げるProjectの一環として作られたものであり、Plkadot
の Project
の一つである。これを用いてスケーラビリティと相互交換性を担保する新たなブロックチェーンの開発を行う。

\hypertarget{acknowledgement}{%
\chapter{7. Acknowledgement}\label{acknowledgement}}

\hypertarget{references}{%
\section{References}\label{references}}

\begin{enumerate}
\def\labelenumi{\arabic{enumi}.}
\item
  https://github.com/ethereum/wiki/wiki/Patricia-Tree
\item
\end{enumerate}
